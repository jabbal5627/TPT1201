\documentclass[a4paper, 12pt]{article}
\usepackage{array}
\usepackage[top=1in, bottom=1in, left=0.7in, right=0.7in]{geometry}
\usepackage{graphicx}

\begin{document}
\setcounter{page}{1}
\begin{tabular}{| m{12em} | m{28em} |} 

\hline

Paper Title & A hybrid brain interface for a humanoid robot assistant \\
\hline

Author(s) & Andrea Finke, Andreas Knoblauch, Hendrik Koesling and Helge Ritter \\
\hline

Abstract/Summary & The representation of relations between handicapped people and robots. How does a robot can help a person which is naturally or by accident type of handicapp. From creation of interface to the experimentation period, this paper will give explanation on how does this relations work. By using the combination of joint movement such as turning, stepping, moving and grasping; with ERD system plus P300 system, the boundary between normal person and handicapped person are shorten. \\
\hline

Problem Solved 
& Experimentation, Creation and Testing of interface so that \\ 
& semi-autonomous robotic personal assistant can help handicapped people. \\
\hline

Claimed Contributions &
The Authors claimed they succesfully implement their ideas in relations with their robots. The total amount of tested actions compare with their minimal are satisfying. According to their report too, the total of error detetion rates are only 0.23 or 23percent. Considering the complexity of the robots and human settings. The experiment was considered a success.  \\
\hline

Related work & 
Hybrid Brain Interface for a humanoid robot assistant. The research article was elaborated 
and comprised of 6 subtopics which consist from the introduction of handicapped problem to 
the system explanations, testing and experiments, results feedback and conclusion. Which 
will be further explain in 4 category which is Introduction, System Construction, 
Experimentation and Result, and Conclusion. \\
& \\
& I.Introduction\\
& The main key of problem that was told inside the articles is about people with motor 
control defcits. These people need daily assistance to complete their daily activity. The 
assistance needed are vary depending on their own type of impairment, such as utilities to 
help speech communication or a wheelchair for area mobility. The solutions ? By using 
robots. Robots are highly valuable and rapidly growing in current world technology. Robots 
are basically the mirror of humans; with fully autonomous system. But Humans prefer that 
robots are not retain fully like humans and have certain degree of optional need to be 
controlled over the tehnical system. [1] One or more inputs modalities are needed. This 
automatically increase the difficulty of impairment patient because approximately all 
possible input channels require some basic motor action (eg:- talking, inputs). The 
authors propose of using a purely cortical signals - which is recorded, translated and 
processed by an EEG-based Brain Machine interface (BMI) that can provides input channel 
independently by depending on the type of a person motor impairment. 
\\

\hline
& This can be used to 
control an autonomous humanoid robot. The Movement state of the robots are purely depends on either Process-directed(continous 
process) or target-directed strategy(exchanging target process).[2] \\

& \\
& II.System Construction \\
& The Authors successfully develop an EEG-based, hybrid BMI with a highly modular structure 
to account for the needs of brain-robot interfaces, in particular to allow for easily 
interfacing to different robots. They used Honda's Humanoid Research Robot. Their BMI system exploits two distinct cortical activity patterns to increase the number 
of controlled dimensions.Raw data stream are constantly scanned for both patterns.\\
& \\
& III.Experimentation and Result \\
& All experiment was conducted with the real, physical robot. The authors used realistic-
type task. They created a simplified "store" in lab which each "store" have 3 shelves, 
arranged at the three sides of a rectangular area. Ten basket in different colors are 
placed on these shelfs. The robot are saved with an image of how the positioning of the 
shelves. The BMI user need to collect five out of ten baskets by navigating the robot 
close to that item. The location are fixed equally to each of the subject. 
There are only four directions that can be used and navigated for the robot which is 
forward, backward and sidesteps left or right side.  There are seven participants involved in this 
experiment and they need to do six times of valid sessions to get six data sets.  Once a robot finish "step" or "turn", the 
system will request new state decision from the data processing module. The robot will 
only leave BMI zone if they are in the vicinity of basket and calling for the movement of 
"grasping". \\

& \\
& After everything is recorded, the author notice of slight EEG error in hybrid BMI, which 
is error detection rate of 0.23; because the EEG in BMI are not perfect. Some participants 
are having difficulty with performing ERD in the complex setting. This assumptions are 
made because of error-state detection rates which is 0.3; for the ERD part alone, but the 
P300 are not affected by this.  \\
& \\
& IV. Conclusion\\
& The Authors implement this advanced yet sophisticated system approach is to help 
handicapped people with robots as their intermediary. . Empirical study was used together with HHRR in the robotics lab where they employed and 
every-day task like shopping and related to test the functionality of the systems.\\

& Out of all participants, only one successfully used the switching of P300 and ERD 
naturally. \\
\hline

Methodology & 
The method used is Empirical. It is proper to use Empirical rather than Theoretical is because handicapped people are physically disable. The statistical data can be proved by the multiple experiment which the the Author had done. \\
& \\
& \includegraphics[scale=1]{image1.png} \\
& \includegraphics[scale=1]{image2.png}\\

\hline

Conclusions & 
The research paper has significantly helped the progress of helping handicapped people with the relation of robots. By using advance technology combination like BMI and HHRR. It started from identification of problems that might be faced by handicapped people. From there they create a system or configuration that can satisfy the needs of motor deficit people and multiple experimentation in a required daily activity.   \\

\hline

What did you learn &
What I learn is a lot, there are many calculations and logics \\
(algorithms/experiments & involve while doing this experiments. The Authors test \\
details)? & their experiment by doing Empirical Method. By using a  \\
& live repetition experiment and multiple of test period, They actually come out with satisfying results. \\
Possible extension/Future & The experiment and project have a very bright future. Because humans will always have disability or been  \\
work & handicapped whether naturally or by accidents. \\

\hline

References & [1] K. Dautenhahn, S. Woods, C. Kaouri, Walters, K. M.L., K.L., and
I. Werry, “What is a robot companion - friend, assistant or butler,” in
IEEE/RSJ Int Conf on Intelligent Robots and Systems (IROS 2005),
2005, pp. 1192–1197. \\
&\\ 
& [2] J. R. Wolpaw, “Brain-computer interfaces as new brain output pathways,”
J Physiol, vol. 579, no. 3, pp. 613–619, March 2007. \\
&\\

\hline
\end{tabular}
\end{document}
