\documentclass[a4paper,12pt]{article}
\usepackage{graphicx}
\usepackage[a4paper]{geometry}
\usepackage{fancyhdr}
\pagestyle{fancy}
\usepackage{array}
\usepackage{apacite}
\usepackage{mathptmx}

\title{A Kinect-Based Gesture Command Control Method
for Human Action Imitations of Humanoid Robots}
\author{\\ \\ \textbf{Ing-Jr Ding, Che-Wei Chang and Chang-Jyun He}\\Department of Electrical Engineering, \\ National Formosa University, \\ Taiwan \\ \url{Ingjr@nfu.edu.tw}}
\date{\today}

\begin{document}
\thispagestyle{empty}
\newgeometry{top=1mm,left=2cm}
\begin{center}

\includegraphics[width=15cm]{mmu.png}
\break\break\break\break
\Huge
TPT 1201\break\break
RESEARCH METHODOLOGY \break
IN \break
COMPUTER SCIENCE \break\break\break\break
\textless ASSIGNMENT 2\textgreater \break\break\break\break 
\Large
PREPARE BY \break\break
\begin{tabular}{| m{7.5cm} | m{9cm} |}
\hline
\hspace{2.3cm}Student ID &\hspace{2.5cm}Student Name \\
\hline
\hspace{2.05cm}1111111118 &\hspace{1.5cm} Jabbal Shafek Affandi \\
\hline
\end{tabular}
\break\break
LECTURE SECTION \hspace{5mm}: TC01\break
TUTORIAL SECTION \hspace{2mm}: TT01\break
LECTURER: {DR. POO KUAN HOONG}
\end{center}

\bibliographystyle{apacite}
\lhead{Jabbal Shafek Affandi 111 111 111 8}
\rhead{TPT 1201}

\restoregeometry
\maketitle
\begin{center}
\bfseries Executive Summary 
\end {center}
{The authors of this paper, Ing-Jr Ding, Che-Wei Chang and Chang-Jyun He; are creating and researching about how a robot can imitate the movement of a human. But the problem lies on how to make the robot smart enough to follow and copy the movement. How it going to move and what gesture can it make ? The researchers came out with a solution; by using Microsoft Kinect System. With the advancement of robotic technology and gesture interface, the researchers wanted to make a robot that can imitate and resembles like a human so that the robot can be of help towards humanity. To get things started, the researchers had done multiple research work which involve the combination of theoretical and empirical method, but inclined more towards empirical style. There was a few errors, implication and problems arise in the progress such as time delay, hardware inefficiency, incorrect gesture command recognition and joint problem. Although the experiment had it flaws, the work and the outcome are still considered as a success. This is because according to the researchers, the number of movement successfully generated by the robot are still quite plenty. The significance of the experiment output are important; because based on the output, they can conclude the statistic and response of the robots and improve more action time with less delay reaction.}
\thispagestyle{empty}
\tableofcontents
\thispagestyle{empty}

\newpage
\setcounter{page}{1}
\section{Introduction} 
\hspace*{1cm} Kinect-Based Gesture Command Control Method for Human Action Imitations of Humanoid Robots or ( KBGHR ) is a research which help improve the robots capability of imitating human together with their gesture. Kinect-Based Robots also used 3 combination of recognition command which is Dynamic Time Warping (DTW) , Hidden Markov Model (HMM) and Principal Component Analysis (PCA)-based eigenspace methods. A humanoid skeleton called Type-A Bioloid are used as test subject in this experiment. Although there are Type-B and Type-C, Type-A are more suitable. A software named RoboPlus was developed for the purpose of this experiment.
\\ \\ \\   
\section{Justification of Research}
\hspace{1cm} KGBHR is an experiment and work which is highly valuable and important. The movement of human can be achieved by using few interactive and recognition. The KGBHR use the 3 main recognition schemes to help the robot imitate a human movement. By also using a combination of Kinect with these 3 recognition schemes, the robot can easily follow the movement of a human with actual response time. The Kinect are used as a control manipulator for the robots. KGBHR will help lots of companies in terms of increasing the robots action time and response output.
\\ \\ \\
\section{Research Objectives}
\hspace*{1cm} This research are meant to accomplish and research more towards : \break 1. To successfully imitate the movement of human and incorporate it into a robot. Such as movement of jumping, walking or even hand and leg movement. 
\break
2. To successfully intergrate Microsoft Kinect with humanoid robots actions imitation  application.
\break
3. To create more efficient and realistic movement of robots.

\newpage
\section{Literature Review}
\hspace*{1cm} The authors showed that they successfully combine Kinect with few recognition schemes to make a robot smart enough to imitate a human movement. Beside that, the authors also showed the problems and specifics guide on does they calculate and programmed the robot. Furthermore, the authors also provide real-life examples to prove to the world that by using Kinect, the reaction time and response value had increased the success rate of humanoid robot imitation. This research result will greatly improve the robotic technology.

\section{Research Methodology}
\hspace*{1cm} The KBGHR techniques and method are effective and efficient in determine the movement imitation. There are two main factors for the performances of humanoid robot imitations, one of it is the recognition accuracy of gesture recognition method to dictate the robot and the other is the matched degree between joint number and joint distribution in the Kinect-captured human skeleton in Bioloid humanoid robot. The authors also had created a software name RoboPlus to keep track the movement of the robot. The gesture recognition method will have a straight influences on the success rate and correctness of  the humanoid robot actions with humanoid skeletons.
\break\break
\includegraphics[width=15cm, height=10cm]{robot1.png}
\break
\includegraphics[width=15cm]{result1.png}
\break
\includegraphics[width=15cm]{result2.png}
\break
\includegraphics[width=15cm]{result3.png}
\break

The figure above shown the result of a Type-A Bioloid robot imitate a human movement. But there are still errors happened during the process of experiment; this is because the total marked joint that Type-A Bioloid had is 18 while Kinect skeleton had 20 marked joint. This result had made the researchers result quite different than intended, but satisfying.

\section{References/Bibliography}
\hspace{1cm}In this paper, we have learned the context and functions of Kinect-based of humanoid robot human movement imitation. Furthermore, we also known that robots will one day help human in many field such as medics, construction, agriculture and others. Beside that, although the field of robotics are still new, the progress and speed of this area are tremendously developing in the world. And day-by-day this robotics can will replace and ease humanity in daily life. Therefore, we can say that this research will surely beneficial to humankind. Still, it might be decades before Kinect-Based Gesture of Human movement Imitation can perfectly work.


\end{document}